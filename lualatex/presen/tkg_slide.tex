% !TeX root = presen.tex
\documentclass[25pt, landscape,oneside]{foils}



% 4:3のスクリーン(古い建物)の場合
%\usepackage[top=15truemm,bottom=22truemm,left=15truemm,right=15truemm,paperwidth=300truemm,paperheight=225truemm]{geometry}

% 16:9のスクリーン(8号館,23号館)の場合
\usepackage[top=15truemm,bottom=22truemm,left=15truemm,right=15truemm,paperwidth=320truemm,paperheight=180truemm]{geometry}


% 箇条書き環境の余白設定
\usepackage[shortlabels]{enumitem}
\setlist[description]{topsep=3mm,parsep=0mm,partopsep=0mm,itemsep=.5\zh,leftmargin=2\zw,labelsep=.5\zw}
\setlist[enumerate]{topsep=3mm,parsep=0mm,partopsep=0mm,itemsep=.5\zh,leftmargin=2\zw,labelsep=.5\zw}
\setlist[itemize]{topsep=3mm,parsep=0mm,partopsep=0mm,itemsep=.5\zh,leftmargin=2\zw,labelsep=.5\zw}


%%%%%%%%%%%%%%%%%%
% フォントの設定
%%%%%%%%%%%%%%%%%%
\usepackage[no-math,deluxe,expert,haranoaji]{luatexja-preset}

\iftrue % 全体的に細身のフォントが良いとき
%\iffalse  % 太めのフォントが良いとき
\setmainfont[BoldFont=HaranoAjiMincho-Bold]{HaranoAjiMincho-Light}
\setsansfont[BoldFont=HaranoAjiGothic-Medium]{HaranoAjiGothic-Light}
\setmainjfont[BoldFont=HaranoAjiMincho-Bold]{HaranoAjiMincho-Light}
\setsansjfont[BoldFont=HaranoAjiGothic-Medium]{HaranoAjiGothic-Light}
\else
\setmainfont[BoldFont=HaranoAjiMincho-Bold]{HaranoAjiMincho-Regular}
\setsansfont[BoldFont=HaranoAjiGothic-Bold]{HaranoAjiGothic-Regular}
\setmainjfont[BoldFont=HaranoAjiMincho-Bold]{HaranoAjiMincho-Regular}
\setsansjfont[BoldFont=HaranoAjiGothic-Bold]{HaranoAjiGothic-Regular}
\fi

\setmonofont{Inconsolata}  % urlやverbatim,listingsなどの欧文フォント

\renewcommand{\kanjifamilydefault}{\gtdefault}  % 日本語フォントをゴシックに
\renewcommand*\familydefault{\sfdefault}
%\mathversion{bold} % 数式フォントを太字に変更する

\usepackage{alltt,upquote,textcomp} % シングル,バッククォートなどの視認性をアップ!
\def\textasciigrave{\char0}

% このパッケージを入れないと verbatim の日本語が何故かゴシックにならない
\usepackage{verbatim}% 


%%%%%%%%%%%%%%%%%%
% PDFの設定
%%%%%%%%%%%%%%%%%%

\usepackage[%
pdfstartview={FitH -32768},%    描画領域の幅に合わせる
bookmarks=true,%                しおり付き
bookmarksnumbered=true,%        章や節の番号をふる
bookmarkstype=toc,%             目次情報のファイル.tocを参照
colorlinks=true,%              ハイパーリンクを色枠に
linkcolor=black,%
urlcolor=black,%
citecolor=black,%
filecolor=black,%
menucolor=black,%
pagecolor=black,%
pdftitle={プロジェクトデザイン3},
pdfsubject={},
pdfauthor={鷹合研究室},
pdfkeywords={},
dvipdfmx-outline-open %%%%%%%%%%%%%%% これがあるとBookmarkを自動的に展開されるみたい・・・
]{hyperref}


%%%%%%%%%%%%%%%%%%%%%%%%%%%
%
% ここから下は必要がない限り書き換えない
%
%%%%%%%%%%%%%%%%%%%%%%%%%%

%%%%%%%%%%%% ブックマークを表示
\usepackage{bookmark}

% スライドの見出しの位置
\setlength{\foilheadskip}{-14mm}

% スライドなので字下げしない
\setlength{\parindent}{0mm}


% ルビ
\usepackage{luatexja-ruby}

\usepackage{bbding} % \PencilRightDown 鉛筆マーク
\usepackage{tikz}
\usepackage[symbol]{footmisc}
\usepackage{xcolor,listings}
\usepackage{multicol}
\usepackage{xurl}
\usepackage{graphicx}
\usepackage{epsfig}

\usepackage{lastpage}
\usepackage{ascmac}
%\usepackage{fancybox,fancyvrb,okumacro}

\renewcommand{\lstlistingname}{List} 

\lstset{%
	language={Python}, 
	backgroundcolor={\color[gray]{.95}},%
	basicstyle={\ttfamily\small},%
	identifierstyle={\ttfamily\small},
	commentstyle={\ttfamily\small\color{red}},
	keywordstyle={\ttfamily\small\color{blue}},
	ndkeywordstyle={},%
	stringstyle={\ttfamily\small\color[rgb]{0,0.5,0}},
	frame={tb},
	breaklines=true,
	columns=[l]{fullflexible},
	%	columns=[l]{fixed},% fixed だと開きすぎ
	basewidth=0.5em,   % これないと行頭のスペースが揃わない
	numbers=left,%
	xrightmargin=0\zw,%
	xleftmargin=3\zw,%
	numberstyle={\ttfamily\small},%
	tabsize=3,
	stepnumber=1, 
	numbersep=0.75\zw,%
	lineskip=-0.3ex,%
	belowcaptionskip=3pt,   % これないと見出しとリスト本体に隙間が毎回変わる
	abovecaptionskip=0pt,
	captionpos=t,
	showstringspaces=false, % 半角スペースを記号で表示しない
}
%

%%%%%%%%%%%% fboxの線幅
\setlength{\fboxrule}{1.5pt}



%%%%%%%%%%%% ページフッタの設定 %%%%%%%%%%%%%
\usepackage{fancyhdr}
\pagestyle{fancy}
\fancyhf{}  % これを入れないと2ページのフッタがずれる
\renewcommand{\headrulewidth}{0pt} % 水平線を消去
\renewcommand{\footrulewidth}{0pt} % 水平線を消去
\renewcommand{\thefootnote}{\fnsymbol{footnote}}

%%%%%%%%%%%%%%%%%%%%%%%%%%%%%%%%%%%%%%
% Verbatim環境にデフォルト値をセット
\usepackage{fancyvrb}
\newenvironment{verbatimx}%
{\small\Verbatim[frame=single,obeytabs,baselinestretch=.65,commandchars=\\\{\}]}%
{\endVerbatim}%

%%%%%%%%%%%%%%%%%%%%%%%%%%%%%%%%%%%%%%
% Verbatim環境にデフォルト値をセット
\newenvironment{myVerbatim}%
{\vspace{-8mm}\Verbatim[frame=single,obeytabs,framesep=2mm,baselinestretch=.7,commandchars=\\\{\}]}%
{\endVerbatim}%


%%%%%%%%%%%%%%%%%%%%%%%%%%%%%%%%%%%%%%
\usepackage{fontawesome5}
\renewcommand{\refname}{文献} % 文献の表記を変更