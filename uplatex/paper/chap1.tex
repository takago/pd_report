% !TeX root = paper.tex


\chapter{序論}

○○○○ ○○○○○○○○○○○○○○○○○○○○○○○○○○○○○○○○○○○○○○○○○○○○○○○○○○○○○○○○○○ ○○○○○○○○○○○○○○○○○○○○○○○○○○○○○○○○○○○○○○○○○○○○○○○○○○○○○○○○ ○○○○○○○○○○○○○○○○○○○○○○○○○○○○○○○○○○○○○○○○○○○○○○○○○○○○○○ ○○○○○○○ ○○○○○○○○○○○○○○○○○○○○○○○○○○○○○○○○○○○○○○○○○○○○○○○○○○○○○○○○○○ ○○○○○○○○○○○○○○○○○○○○○○○○○○○○○○○○○○○○○○○○○○○○○○○○○○○○○○○○ ○○○○○○○○○○○○○○○○○○○○○○○○○○○○○○○○○○○○○○○○○○○○○○○○○○○○○○ ○○○
序論には研究の背景と目的を書く.どういう問題があって,現在はどういう手法がとられているなど,先行研究で試みられている手法などを参考文献を交えながら書く.序論の終わりには,各章に何を述べたかを簡潔に記述する.○○○○ ○○○○○○○○○○○○○○○○○○○○○○○○○○○○○○○○○○○○○○○○○○○○○○○○○○○○


○○○○ ○○○○○○○○○○○○○○○○○○○○○○○○ ○○○○○○○○○○○○○○○○○○○○○○○○○○○○○○○○○○○○○○○○○○○○○○○○○○○○○○○○ ○○○○○○○○○○○○○○○○○○○○○○○○○○○○○○○○○○○○○○○○○○○○○○○○○○○○○○○○○○○○○○○○○○○○○○○○○○○○○○○○○○○○
されば,朝には紅顔ありて夕には白骨となれる身なり.すでに無常の風きたりぬれば,即ち二つの眼たちまちに閉じ,一つの息ながく絶えぬれば,紅顔むなしく変じて,桃李の装いを失いぬるときは,六親眷属あつまりて嘆き悲しめども,さらにその甲斐あるべからず.
 さてしもあるべき事ならねばとて,野外に送りて夜半の煙となし果てぬれば,ただ白骨のみぞ残れり.あわれといふも,なかなか疎かなり.されば,人間の儚き事は,老少不定のさかいなれば,誰の人も早く後生の一大事を心にかけて,阿弥陀仏を深く頼み参らせて,念仏申すべきものなり. あなかしこ,あなかしこ.

○○○○ ○○○○○○○○○○○○○○○○○○○○○○○○○○○○○○○○○○○○○○○○○○○○○○○○○○○○○○○○○○ ○○○○○○○○○○○○○○○○○○○○○○○○○○○○○○○○○○○○○○○○○○○○○○○○○○○○○○○○ ○○○○○○○○○○○○○○○○○○○○○○○○○○○○○○○○○○○○○○○○○○○○○○○○○○○○○○ ○○○○○○○○○○○○○○○○○○○○○○○○○○○○○○○○○○○○○○○○○○○○○○○○○○○○
